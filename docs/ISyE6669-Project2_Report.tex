\documentclass[twoside,12pt]{article}

\setcounter{secnumdepth}{5}

\usepackage[nottoc,numbib]{tocbibind}

\usepackage{amsmath,amsfonts,amsthm,fullpage}
\usepackage{amsmath}
\usepackage{amssymb}
\usepackage{listings}
\setlength{\parindent}{0pt}
\usepackage{graphicx}
\usepackage{bm}
\usepackage[section]{placeins}
\usepackage{lipsum} % just for the example
\usepackage{array}

% Use the standard article template.
%
% The geometry package allows for easy page formatting.
\usepackage{geometry}
\geometry{letterpaper}
% Load up special logo commands.
\usepackage{doc}
% Package for formatting URLs.
\usepackage{url}
% Packages and definitions for graphics files.
\usepackage{epstopdf}
\DeclareGraphicsRule{.tif}{png}{.png}{`convert #1 `dirname #1`/`basename #1 .tif`.png}

\def\argmin{\operatornamewithlimits{arg\, min}}
\newcommand{\rbr}[1]{\left(#1\right)}
\newcommand{\cbr}[1]{\left\{#1\right\}}
\newcommand{\Ncal}{\mathcal{N}}
\renewcommand{\familydefault}{\sfdefault}
\newcolumntype{L}{>{\centering\arraybackslash}m{3cm}}

%
% Set the title, author, and date.
%
\title{ISyE6669, Fall 2016 - Team Project 2 Report }
\author{Ajay D'Souza (adsouza31)}
\author{
  D'Souza, Ajay\\
  \texttt{ajaydsouza@gatech.edu}
  \and
  Huestis, Sarah\\
  \texttt{shuestis3@gatech.edu }
  \and
  Li, Albert\\
  \texttt{albertli@gatech.edu}
  \and
  Yllander, Sean\\
   \texttt{syllander3@gmail.com}
}
\date{}
  
\iffalse
*------------------------------------------------------------*
  These are the instructions for the Final Report
*------------------------------------------------------------*

\fi

\begin{document}

\maketitle
\begin{center}
Project 2 Report
\end{center}

% Add an abstract.
\begin{abstract}
Project 2 Report for ISyE6669 Deterministic Optimization 
\end{abstract}
% Add various lists on new pages.
\pagebreak
\tableofcontents

\pagebreak
\listoffigures
\listoftables

% Start the paper on a new page.
\pagebreak



%
% Body text.
%
\section{1.4 Computation Experiments}
\label{p1}

\subsection{1.4.1 - All Small Subtour Constraints Commented}
\label{q_1_4_1}



$\\$
\subsection{1.4.2 - 1 City Subtour Constraint Active}
\label{q_1_4_2}

\paragraph{a}
The following table $\eqref{tab_q_2_1_3_a}$ tabulates the termination results for the different data files for the Kantorovich formulation in $cs.Kantorovich.partial.mos$. The program had to be terminated only for $cs1.dat$ after 5 mins, the other data files completed successfully.

\begin{table}[h]
\centering
\resizebox{\textwidth}{!}{
	\begin{tabular}{|c|c|c|}
		\hline
 		File & Completed   & Comments \\
		\hline
		cs1.dat & No & Forcibly terminated the run after 300 secs.\\
		\hline
		cs2.dat & Yes & Completed in 2.6 secs.  \\
		\hline
		kant1.dat & Yes & Completed in 0.1 secs. \\
		\hline
		kant2.dat & Yes & Completed in 1.0 secs.  \\
		\hline
	\end{tabular}
	}
	\caption[]{Kantorovich Solver: Termination results for different data files }
	\label{tab_q_2_1_3_a}
\end{table}

\paragraph{b}

The following table $\eqref{tab_q_2_1_3_b}$ tabulates the  branch-and-bound (BB) nodes searched by the BB algorithm when the solver in $cs.Kantorovich.partial.mos$ terminates for the different data files. For cs1.dat the program is had to be terminated after 5 mins,  the other data files completed successfully.

\begin{table}[h]
\centering
\resizebox{\textwidth}{!}{
	\begin{tabular}{|c|c|c|}
		\hline
 		File & branch-and-bound (BB) nodes searched & Comments \\
		\hline
		cs1.dat & 94503 &   Forcibly terminated the run after 300 secs.\\
		\hline
		cs2.dat & 77 &   Completed in 2.6 secs\\
		\hline
		kant1.dat & 1 &   Completed in 0.1 secs. \\
		\hline
		kant2.dat & 1950 &  Completed in 1.0 secs. \\
		\hline
	\end{tabular}
	}
	\caption[]{Kantorovich Solver: branch-and-bound (BB) nodes results searched for different data files }
	\label{tab_q_2_1_3_b}
\end{table}



\paragraph{c}
The following table $\eqref{tab_q_2_1_3_c}$ tabulates the objective value of the best lower bound $Z_L$ and the objective value of the best integer solution $Z_U$ when the solver in $cs.Kantorovich.partial.mos$ terminates for the different files. For cs1.dat the program is has to be terminated after 5 mins,  the other data files completed successfully.

\begin{table}[h]
\centering
\resizebox{\textwidth}{!}{
	\begin{tabular}{|c|c|c|c|}
		\hline
 		File & $Z_L$  & $Z_U$ & Comments \\
		\hline
		cs1.dat & 542 & 544 & Forcibly terminated the run after 300 secs. \\
		\hline
		cs2.dat & 201 & 201 & Completed in 2.6 secs \\
		\hline
		kant1.dat & 22 & 22 & Completed in 0.1 secs. \\
		\hline
		kant2.dat & 30 & 30 & Completed in 1.0 secs. \\
		\hline
	\end{tabular}
	}
	\caption[]{Kantorovich Solver:  $Z_L$ and $Z_U$ results for different data files }
	\label{tab_q_2_1_3_c}
\end{table}

\paragraph{d}
The following table $\eqref{tab_q_2_1_3_d}$ tabulates the optimality gap, i.e. $\frac{(Z_U −Z_L)}{Z_U}$ when the solver in $cs.Kantorovich.partial.mos$ terminates for the different files. For cs1.dat the program is has to be terminated after 5 mins,  the other data files completed successfully.

\begin{table}[h]
\centering
\resizebox{\textwidth}{!}{
	\begin{tabular}{|c|c|c|}
		\hline
 		File & Optimality Gap $\frac{(Z_U −Z_L)}{Z_U}$  & Comments \\
		\hline
		cs1.dat & $0.3677\%$ &  Forcibly terminated the run after 300 secs.  \\
		\hline
		cs2.dat & $0\%$ & Completed in 2.6 secs  \\
		\hline
		kant1.dat & $0\%$ &  Completed in 0.1 secs. \\
		\hline
		kant2.dat & $0\%$ &  Completed in 1.0 secs. \\
		\hline
	\end{tabular}
	}
	\caption[]{Kantorovich Solver: Optimality Gap $\frac{(Z_U −Z_L)}{Z_U}$  for different data files }
	\label{tab_q_2_1_3_d}
\end{table}


\paragraph{e}
The following table $\eqref{tab_q_2_1_3_e}$ tabulates how many integer solutions are found when the solver in $cs.Kantorovich.partial.mos$ terminates for the different files. For cs1.dat the program is has to be terminated after 5 mins,  the other data files completed successfully.

\begin{table}[h]
\centering
\resizebox{\textwidth}{!}{
	\begin{tabular}{|c|L|L|L|L|}
		\hline
 		File &  Count of integer solutions  $count(y_i*x_{ij}) \text{ where } y_i*x_{ij} = ceil(y_i*x_{ij}) $& Total Solutions = $count(y_i*x_{ij})
        $ & Min. No of Rolls integer constraints on both $y_i,x_{ij}$ relaxed  & Min. No Rolls integer constraints on only $x_{ij}$ relaxed  \\
		\hline
		cs1.dat & 79  &  545 & 542 & 542 \\
		\hline
		cs2.dat & 16  &  209 & 200.18 & 201\\
		\hline
		kant1.dat & 8 & 23 & 21.667 & 22 \\
		\hline
		kant2.dat & 17 & 32 & 30 & 30 \\
		\hline
	\end{tabular}
	}
	\caption[]{Kantorovich Solver: How many integer solutions found for different data files }
	\label{tab_q_2_1_3_e}
\end{table}

\FloatBarrier

$\\\\$
\subsection{1.4.3 - 2 City Subtour Constraints Active}
\label{q_1_4_3}


$\\\\$
\subsection{1.4.4 - 3 City Subtour Constraints Active}
\label{q_1_4_4}


$\\\\$
\subsection{1.4.5 - 4 City Subtour Constraints Active}
\label{q_1_4_5}


$\\\\$
\subsection{1.4.6 - Number of Constrains for each formulation}
\label{q_1_4_6}

Table $\eqref{tab_q_1_4_6_1}$ tabulates the results of the time taken in seconds, the number of constraints generated and the optimal tour distance for the 24 city tour for each of the formulations with different initial sub-tour constraints
\begin{table}[h]
\centering
\resizebox{\textwidth}{!}{
	\begin{tabular}{|c|c|c|c|c|}
		\hline
 		Model &  Time Taken & No of Constraints Generated  & Optimal Distance &  \\
		\hline
		No City Constraint & 542.5 & 200.18 & 21.667 & 30 \\
		\hline
		1 City Sub-tour Constraint & 544 & 205 & 22 & 31 \\
		\hline
		2 City Sub-tour Constraint & 543 & 203 & 22 & 31 \\
		\hline
		3 City Sub-tour Constraint & 543 & 203 & 22 & 31 \\
		\hline
		4 City Sub-tour Constraint & 543 & 203 & 22 & 31 \\
		\hline
	\end{tabular}
	}
	\caption[]{Compare results from formulations with the Initial Constraints for a 48 state tour}
	\label{tab_q_1_4_6_1}
\end{table}


$\\\\$
\subsection{1.4.7 - TSP Tour plots}
\label{q_1_4_7}

\begin{figure}[!htbp]
\centering
Figure $\eqref{hw5_q1_4_1}$ is the plot of the feasible regions of the XOR statement. The feasible region could be either $S_1 $ or $S_2$ but not both at the same time.
 \includegraphics[scale=8]{hw5_q1_4_1} 
\caption{Feasible Region of the XOR statement $S_1 $ or $S_2$}
\label{hw5_q1_4_1}
\end{figure}
\FloatBarrier





$\\\\$
\subsection{1.4.8 - TSP Tour with City Names}
\label{q_1_4_8}


$\\\\$
\subsection{1.4.9 - TSP for 24 City Tour}
\label{q_1_4_9}

The following are the 24 cities randomly chosen for the 24 city tour
\begin{verbatim}
coord  : [
1 6898 1885
3 5530 1424
4 401 841
5 3082 1644
6 7608 4458
8 7265 1268
11 5468 2606
15 6347 2683
16 6107 669
19 7732 4723
20 5900 3561
21 4483 3369
23 5199 2182
26 675 1006
30 7352 4506
31 7545 2801
34 4608 1198
35 23 2216
38 7392 2244
40 6271 2135
43 7280 4899
44 7509 3239
47 5185 3258
48 3023 1942]
\end{verbatim}

Table $\eqref{tab_q_1_4_9_1}$ tabulates the results of the time taken in seconds, the number of constraints generated and the optimal tour distance for the 24 city tour for each of the formulations with different initial sub-tour constraints
\begin{table}[h]
\centering
\resizebox{\textwidth}{!}{
	\begin{tabular}{|c|c|c|c|c|}
		\hline
 		Model &  Time Taken & No of Constraints Generated  & Optimal Distance &  \\
		\hline
		No City Constraint & 542.5 & 200.18 & 21.667 & 30 \\
		\hline
		1 City Sub-tour Constraint & 544 & 205 & 22 & 31 \\
		\hline
		2 City Sub-tour Constraint & 543 & 203 & 22 & 31 \\
		\hline
		3 City Sub-tour Constraint & 543 & 203 & 22 & 31 \\
		\hline
		4 City Sub-tour Constraint & 543 & 203 & 22 & 31 \\
		\hline
	\end{tabular}
	}
	\caption[]{Compare results from formulations with the Initial Constraints for a 24 state tour}
	\label{tab_q_1_4_9_1}
\end{table}


$\\\\$
\section{Source Code}

\subsection{TSP-DFJ-partial.mos}
\begin{verbatim*}
model ModelName
uses "mmxprs"; !gain access to the Xpress-Optimizer solver
uses "mmsystem" ! include package to operating systems

N := 48  ! number of cities

declarations
	Cities = 1 .. N                         	! set of cities
	coord: array(Cities,1..3) of real			! array of coordinates of cities, to be read from US48.dat
	dist: array(Cities,Cities) of real  		! distance between each pair of cities
	x : array(Cities,Cities) of mpvar       	! decision variables
	flag : integer                          	! flag=0: not optimal yet; flag=1: optimal 
	ind : range                             	! dynamic range
	numSubtour : integer                    	! number of generated subtours
	numSubtourCities : integer					! number of cities on a generated subtour
	SubtourCities : array(Cities) of integer	! SubtourCities(i)=1 means city i is on the subtour 
	subtourCtr : dynamic array(ind) of linctr   ! dynamic array of subtour elimination constraints
	TotalDist : linctr    						! objective constraint
	
	! constraint for only one path out of each city
	leavingConstraint: array(Cities)  of linctr      
	! constraint for only one path into of each city
	enteringConstraint: array(Cities)  of linctr     
	! constraints for preventing one city subtour
	oneCitySubTourConstraint: array(Cities)  of linctr     
	! constraints for preventing two city subtour
	twoCitySubTourConstraint: dynamic array(range)  of linctr     
	! constraints for preventing three city subtour
	threeCitySubTourConstraint: dynamic array(range)  of linctr  
	! constraints for preventing four city subtour
	fourCitySubTourConstraint: dynamic array(range)  of linctr  
	
	! constraint for a TSP tour to have N edges
	!tspConstr: linctr
	
	! counter for dynamic arrays
	cons: integer          
	
	! time variables
	starttime: real
	 
	! keep track of next city for each city
	nextCity: array(Cities) of integer
	
	! keep the set of cities in the subtour, used to get the smallest subtour
	! in a aolution
	smallestSubTourSet, allSubTourCitiesSet, new_tour: set of integer
	
end-declarations

!!!!!!!!!!!!!!!!!!!!!!!!!!!!!!!!!!!!!!!!!!!!!!!!!!!!!!!!!!!!!!!!!!!!!
!!!!!  save the tour to output file for plotting !!!!!!!!!!!!!!!!!!!!
!!!!!!!!!!!!!!!!!!!!!!!!!!!!!!!!!!!!!!!!!!!!!!!!!!!!!!!!!!!!!!!!!!!!!

! record initial time
starttime:=gettime 

fopen("l_US"+N+".output",F_OUTPUT)
writeln("Starting at time :",starttime)
fclose(F_OUTPUT)

! initialization part is given 
initializations from "US"+N+".dat"
     coord
end-initializations

! compute dist(i,j) the distance between each pair of cities using (x,y) 
! coordinates of the cities, which are in the array coord
! you may need square root function sqrt()
!!!!!!!!!!!!!!! fill in your code here !!!!!!!!!!!!!!!!!!!! 
forall ( i in Cities ) do
	forall ( j in Cities ) do
		if ( i = j) then
			dist(i,j) := 0.0
		else
			dist(i,j) := sqrt( (coord(i,2)-coord(j,2))^2 + (coord(i,3)-coord(j,3))^2 )
			dist(j,i) := dist(i,j)
		end-if
	end-do
end-do

!!!!!!!!!!!!! objective: total distance of a tour
!!!!!!!!! fill in your code here !!!!!!!!!!!!!!!
TotalDist := sum(i in Cities, j in Cities ) x(i,j)*dist(i,j)

!!!!!!!!!! write constraint: x(i,j) is binary !!!!!!!!!!!!!!!!
!!!!!!!!! fill in your code here !!!!!!!!!!!!!!!
forall(i in Cities, j in Cities ) do
	x(i,j) is_binary
end-do

!!!!!!!!!!! write assignment constraints: in and out constraints for each city !!!!!!!!!!!!!!!!!!!
!!!!!!!!! fill in your code here !!!!!!!!!!!!!!!
forall(i in Cities) do
 leavingConstraint(i) := sum ( j in Cities ) x(i,j) = 1 
end-do

forall(j in Cities) do
 enteringConstraint(j) := sum ( i in Cities ) x(i,j) = 1 
end-do

! 1.
!!!!!!!!!! write 1-city subtour elimination constraints here !!!!!!!!!!!!!!!!!
!!!!!!!!! fill in your code here !!!!!!!!!!!!!!!
! generate the 48c2 combinations and add the constraint for each
! combination

forall(i in Cities) do
 ! add the no closed loop constraint for each 1 city combination
 oneCitySubTourConstraint(i) := x(i,i) = 0 
end-do


!2.
!!!!!!!!!!! write 2-city subtour elimination constraints here !!!!!!!!!!!!!!!!!!!
!!!!!!!!! fill in your code here !!!!!!!!!!!!!!!
cons := 1
! generate the 48c2 combinations and add the constraint for each
! combination
forall(i in Cities, j in Cities) do
 
 if (i>=j) then
   next
 end-if
  
 create(twoCitySubTourConstraint(cons))
 
 ! add the no closed loop constraint for each 2 city combination
 twoCitySubTourConstraint(cons) := x(i,j) + x(j,i ) <= 1
 
 cons += 1
end-do


! 3.
!!!!!!!!!! write 3-city subtour elimination constraints here !!!!!!!!!!!!!!
!!!!!!!!! fill in your code here !!!!!!!!!!!!!!!
(!
cons := 1
! generate the 48c3 combinations and add the constraint for each
! combination
forall(i in Cities, j in Cities, k in Cities ) do
 
 if (( i>=j) or ( i>=k) or (j>=k) ) then
 	next
 end-if
  
 create(threeCitySubTourConstraint(cons))
 
 ! add the no closed loop constraint for each 3 city combination
 threeCitySubTourConstraint(cons) :=  x(i,j) + x(i,k ) + x(j,k) +
 										x(j,i) + x(k,i ) + x(k,j) <= 2
 
 cons += 1
end-do

!4.
!!!!!!!!!! write 4-city subtour elimination constraints here !!!!!!!!!!!!!!
!!!!!!!!! fill in your code here !!!!!!!!!!!!!!!
cons := 1

! generate the 48c4 combinations and add the constraint for each
! combination
forall(i in Cities, j in Cities, k in Cities, l in Cities ) do
 
 if (( i>=j) or ( i>=k) or (i>=l) or (j>=k) or ( j>=l) or ( k>=l) ) then
 	next
 end-if
 
 create(fourCitySubTourConstraint(cons))
 
 ! add the no closed loop constraint for each 4 city combination
 fourCitySubTourConstraint(cons) :=  x(i,j) + x(i,k ) + x(i,l) + 
  										x(j,i) + x(k,i ) + x(l,i) + 
  										x(j,k) + x(j,l) +
  										x(k,j) + x(l,j) +
  										x(k,l) +
  										x(l,k) <= 3
  										
  cons += 1
end-do
!)

!!!!!!!!!!!!! constraint generation algorithm !!!!!!!!!!!!!!!!!!!!!!!!!!!!!!!!
numSubtour := 0   ! number of added subtour elimination constraints is zero
flag := 0 ! initalize flag to be 0, so no optimal solution has been found yet

repeat 
	
	!!!!!!!!!!!!!!! Solve the restricted master problem  !!!!!!!!!!!
	minimize(TotalDist)
	
	! Output the solution of the restricted master problem
	writeln("The restricted master problem is solved:")
	forall (i in Cities, j in Cities) do
		if abs(getsol(x(i,j))-1)<0.1 then  
		! note here we could have simply written "if getsol(x(i,j))=1 then", 
		! but I found cases where Xpress doesn't output all such x(i,j)'s. 
		! So this is a quick and ugly fix. 
		! You can use this trick in the later part when you need to check if x(i,j) is 1 or not
		! Also, feel free to develop your own solution
			writeln("x(",i,",",j,")=",getsol(x(i,j)))
		 ! save the next city information for each city in array
			next_city(i) := j
		end-if
	end-do
	
	!!!!!!!!!!!!!!!!!!!!!  find a subtour !!!!!!!!!!!!!!!!!!!!!!!!!!!!!!!!!!!!!!!!!!!!!!!!!!!!!!!!!!!!!!!!!
	! We want to find a subtour starting at city 1 (Atlanta) and ending at City 1 (such a subtour always exists!)
	! First, initialize a few things:
	numSubtourCities := 0    ! the number of cities on the subtour
	forall (i in Cities) do  ! SubtourCities(i)=1 if city i is on the subtour, initialize all entries to zero
		SubtourCities(i):=0  ! need to change entries when city i is found on the tour
	end-do
	SubtourCities(1) := 1  ! City 1 (Atlanta) is always on the subtour
		
	! Start the procedure to look for a subtour starting and ending at City 1. 
	! The basic algorithm is discussed in the hand-out
	! Note you need to update SubtourCities for cities that are on the subtour 
	! You also need to keep track of the number of cities numSubtourCities on the subtour
	!!!!!!!!! fill in your code here !!!!!!!!!!!!!!!
	
	! loop from atlanta till we reach atlanta back again
	currentCity := 1
	repeat
		! set of cities in this subtour
		smallestSubTourSet += {currentCity}
		
		! find the next city j for TSP from this city i
		currentCity := next_city(currentCity)
		numSubtourCities += 1
		SubtourCities(currentCity) := 1
				
	until ( currentCity = 1 )
	
	!!!!!!!!!!!!!!!!!!!!!!!!!!!!!!!!!!!!!!!!!!!!!!!!!!!!!!!!!!!!!!!!!!!!!!!!!!!!!!!!!!!!!!!!!!!!!!!!!!!!!!!!!
	
	! output the subtour you found
	writeln("Found a subtour of distance ", getobjval, " and ", numSubtourCities, " cities")
	writeln("Cities on the subtour are:")
	forall (i in Cities | SubtourCities(i) = 1) do
	! Note: forall ( ... | express ) is very useful, you may need to use it in the following 
	! part to add subtour elimination constraints
		write(i, " ")
	end-do
	writeln("")
	
	!!!!!!!!!!!!!!!!!!!!!!!!!!!!!!!!!!!!!!!!!!!!!!!!!!!!!!!!!!!!!!!!!!!!!
	!!!!!  save the tour to output file for plotting !!!!!!!!!!!!!!!!!!!!
	!!!!!!!!!!!!!!!!!!!!!!!!!!!!!!!!!!!!!!!!!!!!!!!!!!!!!!!!!!!!!!!!!!!!!
	fopen("l_US"+N+".output",F_OUTPUT+F_APPEND)
	writeln("Constraints Added : ",numSubtour)
	writeln("Time in Secs : ", gettime-starttime)
	writeln("Objective Distaince : ", getobjval)
	writeln("Subtour from Atlanta : ", numSubtourCities)
	writeln("Full Tour:")
	forall (i in Cities) do
		writeln(i,"\t",next_city(i))
	end-do
	writeln("--------------------------")
	fclose(F_OUTPUT)
	
	!!!!!!!!!!!!!!!!!!!! add the subtour elimination constraint !!!!!!!!!!!!!!!!!!!!!!!!!!!!!!!!!!!!!!!!!!!!!
	! If the subtour found above is indeed a subtour (i.e. has fewer than 48 cities), 
	! then add the corresponding subtour elimination
	!  constraint to the problem
	! otherwise, if the subtour has 48 cities, then it's a TSP tour and optimal, 
	! terminate the constraint generation by setting the flag to 1
	!!!!!! fill in you code !!!!!!!!!!
	
	if ( numSubtourCities = N ) then
	   flag := 1
	else

		! find the smallest subtour in the solution and a constraint to break it
			
		! only if the subtout with city 1 is > 1, else it has to be the smallest size
		if getsize(smallestSubTourSet) > 1 then
			
			! set to keep track of cities in subtours xconsidered so far,
			! initialized to the set of cities in subtour with city 1
			allSubTourCitiesSet := smallestSubTourSet
			
			! go over the cities not in subtours considered so far and find a subtour for each
			forall (i in Cities) do
			
			 ! if the city is not in the subtour considered so far, then find the subtour
			 ! having this city
			 if ( i not in allSubTourCitiesSet) then
				
				! the city is not in any subtour so far, find the subtour with this city
				new_tour := {}
				currentCity := i
				repeat
					new_tour += {currentCity}
				until ( currentCity = i ) 
				
				! add the cities in this subtour to the cities in subtour so far
				allSubTourCitiesSet += new_tour
				
				! if this tour is smaller than the earlier one  then save this as the smallest
				! subtour in this solution
				if ( getsize(new_tour) < getsize(smallestSubTourSet) ) then
					smallestSubTourSet := new_tour
				end-if
				
				! if this smallest subtour is 1 then any subtour cannot be smaller than this
				! subtour in this solution
				if ( getsize(new_tour) = 1 ) then
					smallestSubTourSet := new_tour
					break
				end-if
			 
			 end-if
			 
			end-do
			
		end-if
		
		! add a constraint to break the smallest Sub Tour found 
		numSubtour += 1
		create(subtourCtr(numSubtour))
		subtourCtr(numSubtour):= sum (i in smallestSubTourSet ) x(i, next_city(i)) <= getsize(smallestSubTourSet) - 1

	end-if

    !!!!!!!!!!!!!!!!!!!!!!!!!!!!!!!!!!!!!!!!!!!!!!!!!!!!!!!!!!!!!!!!!!!!!!!!!!!!!!!!!!!!!!!!!!!!!!!!!!!!!!!!!
  
    
until flag = 1

!!!!!!!!!!!!!!!!!!!! end of the constraint generation algorithm !!!!!!!!!!!!!!!!!!!!!!!!!

	writeln("\nOptimal TSP distance = ", getobjval)
	forall (i in Cities, j in Cities) do
		if abs(getsol(x(i,j))-1)<0.1 then
			writeln("x(",i,",",j,")=",getsol(x(i,j)))
		end-if
	end-do
	
	! write the solution to an output file 
	! then run matlab code US48TourPlot.m to plot the tour
	fopen("US"+N+".output",F_OUTPUT)
	forall (i in Cities, j in Cities) do
		if abs(getsol(x(i,j))-1)<0.1 then
			writeln(i,"\t",j)
		end-if
	end-do
	fclose(F_OUTPUT)

writeln("End running model")

end-model
\end{verbatim*}


\subsection{US24TourPlot.m}
\begin{verbatim}
clear all;

outputfile = 'US24.output';
f = fopen(outputfile,'r');
x = fscanf(f, '%d\t %d', [2, inf]);
fclose(f);
x = x';

coordfile = 'US48.input';
f = fopen(coordfile,'r');
coord = fscanf(f, '%d %f %f', [3, inf]);
fclose(f);
coord = coord';

tour = zeros(24,1);
tour(1) = 1;
fromCity = 1;
for k = 1 : 24
    tour(k+1) = x(fromCity,2);
    fromCity = x(fromCity,2);
end
figure(1);
plot(coord(tour(:,1),2),coord(tour(:,1),3),'rO');
hold on;
plot([coord(tour(:,1),2);coord(1,2)],[coord(tour(:,1),3);coord(1,3)]);
plot(coord(1,2),coord(1,3),'r*'); % the star marks Atlanta.
\end{verbatim}



\section{Distribution of Team Effort}
Equal effort by all team members. Each team member implemented the solutions independently and verified the results among the team.




%\addcontentsline{toc}{section}{References}
\bibliographystyle{plain}
% Generate the bibliography.
\begin{thebibliography}{9}

\bibitem{ekstrand_2011}
  Bertsimas Dimitris, Tsitsiklis N. John,
  \emph{Introduction to Linear Optimization},
  Athena Scientific Edition 6,
  1997.

\end{thebibliography}

\end{document}